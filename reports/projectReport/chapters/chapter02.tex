\chapter{Numerics and algorithm} % (fold)
\label{cha:algorithm}

Please keep in mind the meaning of the 'exponents' $\left.\cdot^\circ\right.$ and $\left.\cdot^+\right.$: variable at this timestep and at the next timestep.

\section*{Discretization}

Following discretization was chosen to solve the equations given in \ref{eq:conti}, \ref{eq:RANSmodified}, \ref{eq:keTransport04} and \ref{eq:keTransport05}:
\begin{itemize}
\item spacial discretization: finite difference method
\item time discretization: mixed treatment (pressure implicit with backward Euler; velocity, turbulent kinetic energy and dissipation rate explicit - see Griebel \citep{griebel1998}).
\end{itemize}
The discretization of the two additional transport equations is straight forward, with only two additional derivatives to be discretized, which are identical for both equations (here only presented in x-direction):

\begin{align*}
\left[\abl{}{x}\left(A\,\abl{B}{x}\right)\right]_{i,j,k}
= \frac{2}{\Delta x_{i,j,k}}
\left( 
\frac{(\Delta x_{i+1,j,k}\cdot A_{i,j,k}+\Delta x_{i,j,k}\cdot A_{i+1,j,k}) \cdot (B_{i+1,j,k}-B_{i,j,k})}{(\Delta x_{i+1,j,k}+\Delta x_{i,j,k})^2} \right.\\
-\left.
\frac{(\Delta x_{i,j,k}\cdot A_{i-1,j,k}+\Delta x_{i-1,j,k}\cdot A_{i,j,k}) \cdot (B_{i,j,k}-B_{i-1,j,k})}{(\Delta x_{i,j,k}+\Delta x_{i-1,j,k})^2} 
\right)
\end{align*}

\begin{align*}
\left[\abl{(u\,B)}{x}\right]_{i,j,k}
= 
\frac{1}{\Delta x_{i,j,k}}\left(
u_{i,j,k}\frac{\Delta x_{i+1,j,k}B_{i,j,k}+\Delta x_{i,j,k}B_{i+1,j,k}}{\Delta x_{i+1,j,k}+\Delta x_{i,j,k}}\right.
\\-\left.
u_{i-1,j,k}\frac{\Delta x_{i,j,k}B_{i-1,j,k}+\Delta x_{i-1,j,k}B_{i,j,k}}{\Delta x_{i,j,k}+\Delta x_{i-1,j,k}}
\right)
\end{align*}
with:
\begin{center}
\begin{tabular}{lll}
Equ. \ref{eq:keTransport04}: & $A=B_k$           & $B=k$\\
Equ. \ref{eq:keTransport05}: & $A=B_\varepsilon$ & $B=\varepsilon$
\end{tabular}
\end{center}

\newpage
\section*{Timestep}
A first guess for the new timestep is calculated with the same equation as implemented for the algebraic turbulence model:

\begin{align}
\Delta t < \min\left\lbrace
\nu_{max}^{-1}\cdot
\left( 
\frac{1}{\Delta x^2}+
\frac{1}{\Delta y^2}+
\frac{1}{\Delta z^2}
\right)^{-1}
,
\frac{\Delta x}{|u_{max}|},
\frac{\Delta y}{|v_{max}|},
\frac{\Delta z}{|w_{max}|}
\right\rbrace
\end{align}
with:
\begin{align}
\nu_{max}=\nu_M = 2\cdot\left( \nu+\nu_T \right)
\end{align}
It was examined whether the diffusion terms of the $k$- and $\epsilon$- transport equations are more critical than the one from the momentum equation ($\nu_K$ and $\nu_E$). It has, however, proven that the diffusion rate of the momentum equation ($\nu_m$) is the highest:
\begin{align}
\nu_K = \nu+\frac{\nu_T}{\sigma_k}<\nu_M\\
\nu_E = \nu+\frac{\nu_T}{\sigma_\varepsilon}<\nu_M
\end{align}
The timestep is reduced by so much that in the next timestep there are no negative $k$ and $\varepsilon$\footnote{The user can specify whether a certain error for $k$ and $\varepsilon$ is acceptable or not in the configuration file.}.

\noii Big increases in the timestep were programmatically prevented:
\begin{align*}
\Delta\,t^+=\min{(\Delta\,t^\circ\cdot 1.1,\,\Delta\,t^+_{guess})}
\end{align*} 
for stability reasons.

\newpage
\section*{Algorithm}
Following steps are performed in our program to solve the RANS with the help of the \ke\, model in each timestep (for a simplified graphical representation of the algorithm see figure~\ref{fig:alg}):
\begin{itemize}
\item[1.] Estimate $\Delta t$
\item[2.] Calculate $k$ and $\varepsilon$ (first step):
\begin{align}\label{KK1}
K^\circ&=
\abl{}{x_i}\left(\nu_t\,\abl{k}{x_i}\right)
-
\abl{(u_i\cdot k)}{x_i}
+
\frac{1}{2}\nu_t
(\,S_{i,j} \cdot S_{i,j})
-
\varepsilon
\\\label{EE1}
E^\circ&=\frac{c_\varepsilon}{c_\mu}\cdot
\abl{}{x_i}\left(f_\mu\,\nu_T\,\abl{\varepsilon}{x_i}\right)
-
\abl{(u_i\cdot \varepsilon)}{x_i}
+
\frac{c_1\,f_1}{2}k
(\,S_{i,j} \cdot S_{i,j})
-
c_2\,f_2\frac{\varepsilon^2}{k}
\end{align}\vspace{-0.5cm}



\item[3.] Calculate temporal $k$ and $\varepsilon$ (second step):
\begin{align}
k^* &= k^\circ + \Delta t \cdot K^\circ\\
\varepsilon^* &= \varepsilon^\circ + \Delta t \cdot E^\circ
\end{align}\vspace{-0.5cm}

Check for validity and if necessary adjust timesteps.

\item[4.] Make temporal $k^*$ and $\varepsilon^*$ final ($k^+$ and $\varepsilon^+$)

\item[5.] Calculate $F_i^\circ$:
\begin{align}
F_i^\circ=u_i^\circ+\Delta t\left(-\abl{u_i^\circ\,u_j^\circ}{x_j} +2\,\abl{}{x_j} \left( \,\left(\nu+\nu_T^\circ\right)S_{ij}^\circ\right)+f_i^\circ  \right)
\end{align}
\item[6.] Solve the Pressure-Poission-Equation for $(p^+/\rho)$ $\rightarrow$ \textsc{PETSc}:
\begin{align}
\frac{1}{\Delta t} \abl{F_i^\circ}{x_i}  =
\frac{\partial^2 (p^+/\rho)}{\partial x_i\,\partial x_i}
\end{align}
\item[7.] Calculate velocities $u_i^+$:
\begin{align}
u_i^+
=F_i^\circ-\frac{\Delta t}{\rho}\,\abl{(p^+/\rho)}{x_i}
\end{align}
\item[8.] Calculate eddy viscosity $\nu_T^\circ$:
\begin{align}
\nu_{t}^+=c_\mu\frac{\left(k^+\right)^2}{\varepsilon^+}
\qquad\qquad\text{and}\qquad\qquad
l_m = \sqrt{\frac{3}{2}} c_\mu\frac{\left(k^+\right)^{3/2}}{\varepsilon^+}
\end{align}
\vspace{-0.5cm}
\item[9.] Calculate wall near correction factors ($f_1$, $f_2$, $f_2$, $f_{\mu}$, $D$ and $E$)
\item[10.] Calculate additional useful variables: velocity fluctuation $u'$ and pressure $p$:
\begin{align}
u'=\sqrt{\frac{2}{3}k}\qquad\qquad
\frac{{p}}{\rho}=\frac{{P}}{\rho}-\frac{2}{3}\,K=\frac{{P}}{\rho}-u'u'
\end{align}
\end{itemize}





\clearpage
\begin{center}

\begin{figure}[!htb]

% =================================================
% Set up a few colours
\colorlet{lcfree}{green}
\colorlet{lcnorm}{blue}
\colorlet{lccong}{red}
% -------------------------------------------------
% Set up a new layer for the debugging marks, and make sure it is on
% top
\pgfdeclarelayer{marx}
\pgfsetlayers{main,marx}
% A macro for marking coordinates (specific to the coordinate naming
% scheme used here). Swap the following 2 definitions to deactivate
% marks.
\providecommand{\cmark}[2][]{%
  \begin{pgfonlayer}{marx}
    \node [nmark] at (c#2#1) {#2};
  \end{pgfonlayer}{marx}
  } 
\providecommand{\cmark}[2][]{\relax} 
% -------------------------------------------------
% Start the picture
\begin{tikzpicture}[%
    >=triangle 60,              % Nice arrows; your taste may be different
    start chain=going below,    % General flow is top-to-bottom
    node distance=6mm and 60mm, % Global setup of box spacing
    every join/.style={norm},   % Default linetype for connecting boxes
    ]
% ------------------------------------------------- 
% A few box styles 
% <on chain> *and* <on grid> reduce the need for manual relative
% positioning of nodes
\tikzset{
  base/.style={draw, on chain, on grid, align=center, minimum height=4ex},
  proc/.style={base, rectangle, text width=8em},
  test/.style={base, diamond, aspect=2, text width=5em},
  term/.style={proc, rounded corners},
  % coord node style is used for placing corners of connecting lines
  coord/.style={coordinate, on chain, on grid, node distance=6mm and 25mm},
  % nmark node style is used for coordinate debugging marks
  nmark/.style={draw, cyan, circle, font={\sffamily\bfseries}},
  % -------------------------------------------------
  % Connector line styles for different parts of the diagram
  norm/.style={->, draw, lcnorm},
  free/.style={->, draw, lcfree},
  cong/.style={->, draw, lccong},
  it/.style={font={\small\itshape}}
}
% -------------------------------------------------
% Start by placing the nodes
\node [proc, densely dotted, it] (p0) {input};
% Use join to connect a node to the previous one 
\node [proc, join, below=1.5cm of p0] (KoEo)  {Calc $K^\circ$ and $E^\circ$};
\node [proc, join] (p1) {set $\Delta t$ and $n=5$};

\node [test, join, below=2.5cm of p1] (adap) {$k^*<0$\\$\varepsilon^*<0$\\$n\ge 0$};

\draw[thick,black] ($(adap.north west)+(-2.0,2.5)$)  rectangle ($(adap.south east)+(4.5,-0.8)$);

%\path (adap.east) to node [near start, xshift=2.0em] {$yes$} (adap.north); 
%  \draw [->,lcnorm] (adap.east) -- ($(adap.east) + (+1.0,+0.0)$) -- ($(adap.east) + (1.0,+1.5)$) -| (adap.north);
  
  
\node [proc,above right = +1.0cm and +4.0cm of adap] (adapp)  {$\Delta t /=2$, $n-=1$}; 
  
  
\path (adap.east) to node [near start, xshift=1.0em] {$yes$} (adapp.south); 
  \draw [->,lcnorm] (adap.east)  -| (adapp.south);

\path (adapp.north) to node [near start, xshift=2.0em] {$ $} (adap.north); 
  \draw [->,lcnorm] (adapp.north) -- ($(adapp.north) + (+0.0,+0.4)$)  -| (adap.north);  

\node [proc,below left = 3cm and -3.0cm of adap] (ke1)  {Calc $k^+$ and $\varepsilon^+$};

\node [proc, below left = 3cm and +3.0cm of adap] (fgh1) {Calc $F^\circ_i$};

\path [norm] (adap) |- (fgh1);
\path [norm] (adap) |- (ke1);

\node [proc, join, below=2cm of fgh1] (loop)  {Calc $p^+$};

\draw[black,thick] ($(loop.north west)+(-1.0,0.3)$)  rectangle ($(loop.south east)+(1.0,-0.8)$);

\node [proc, join, below=2cm of loop] (t1) {Calc $u^+$};

\path [norm]  ($(loop.south east) + (-0.3,+0.0)$) -| ($(loop.south east) + (-0.3,-0.5)$) |- ($(loop.south east) + (0.7,-0.5)$) |- (loop);



% No join for exits from test nodes - connections have more complex
% requirements
% We continue until all the blocks are positioned
\node [proc, below left = 2cm and -3.0cm of t1] (p2) {Calc: $\nu_t$, $f_1$, $f_2$, $f_\mu$, $D$, $E$, $t$};


%\node [proc, join] (p3) {Dispatch message};

\path [norm] (ke1) |- (p2);
\path [norm] (t1)  |- (p2);


\node [test, join] (t2) {$t<t_{end}$};

\path (t2.west) to node [near start, xshift=-3em, yshift=-9.0em] {$yes$} (KoEo.north); 
  \draw [->,lcnorm] (t2.west) -- ($(t2.west) + (-5.0,+0.0)$) -- ($(KoEo.north west) + (-5.0,+0.3)$) -| (KoEo.north);

\draw[black,thick] ($(p1.north west)+(-7.0,1.9)$)  rectangle ($(t2.south east)+(6,-0.5)$);

\node [term, join] (p10) {exit};

\node[draw=none,fill=none] at (-1,-9.3) {PETSc};
\node[draw=none,fill=none] at (+3.9,-1.8) {Adaptive timestep};


%\node [test] (t3) {Capacity?};
%\node [test] (t4) {$k \mathbin{{-}{=}} 1$};
% We position the next block explicitly as the first block in the
% second column.  The chain 'comes along with us'. The distance
% between columns has already been defined, so we don't need to
% specify it.
%\node [proc, fill=lcfree!25, right=of p1] (p4) {Reset congestion};
%\node [proc, join=by free] {Set \textsc{mq} wait flag};
%\node [proc, join=by free] (p5) {Dispatch message};
%\node [test, join=by free] (t5) {Got msg?};
%\node [test] (t6) {Capacity?};
% Some more nodes specifically positioned (we could have avoided this,
% but try it and you'll see the result is ugly).
%\node [test] (t7) [right=of t2] {$k \mathbin{{-}{=}} 1$};
%\node [proc, fill=lccong!25, right=of t3] (p8) {Set congestion};
%\node [proc, join=by cong, right=of t4] (p9) {Close queue};

% -------------------------------------------------
% Now we place the coordinate nodes for the connectors with angles, or
% with annotations. We also mark them for debugging.
%\node [coord, right=of t1] (c1)  {}; \cmark{1}   
%\node [coord, right=of t3] (c3)  {}; \cmark{3}   
%\node [coord, right=of t6] (c6)  {}; \cmark{6}   
%\node [coord, right=of t7] (c7)  {}; \cmark{7}   
%\node [coord, left=of t4]  (c4)  {}; \cmark{4}   
%\node [coord, right=of t4] (c4r) {}; \cmark[r]{4}
%\node [coord, left=of t7]  (c5)  {}; \cmark{5}   
% -------------------------------------------------
% A couple of boxes have annotations
%\node [above=0mm of p4, it] {(Queue was empty)};
%\node [above=0mm of p8, it] {(Queue was not empty)};
% -------------------------------------------------
% All the other connections come out of tests and need annotating
% First, the straight north-south connections. In each case, we first
% draw a path with a (consistently positioned) annotation node, then
% we draw the arrow itself.
%\path (t1.south) to node [near start, xshift=1em] {$y$} (p2);
%  \draw [*->,lcnorm] (t1.south) -- (p2);
%\path (t2.south) to node [near start, xshift=1em] {$y$} (t3); 
%  \draw [*->,lcnorm] (t2.south) -- (t3);
%\path (t3.south) to node [near start, xshift=1em] {$y$} (t4); 
%  \draw [*->,lcnorm] (t3.south) -- (t4);
%\path (t5.south) to node [near start, xshift=1em] {$y$} (t6); 
%  \draw [*->,lcfree] (t5.south) -- (t6);
%\path (t6.south) to node [near start, xshift=1em] {$y$} (t7); 
%  \draw [*->,lcfree] (t6.south) -- (t7); 
% ------------------------------------------------- 
% Now the straight east-west connections. To provide consistent
% positioning of the test exit annotations, we have positioned
% coordinates for the vertical part of the connectors. The annotation
% text is positioned on a path to the coordinate, and then the whole
% connector is drawn to its destination box.
%\path (t3.east) to node [near start, yshift=1em] {$n$} (c3); 
%  \draw [o->,lccong] (t3.east) -- (p8);
%\path (t4.east) to node [yshift=-1em] {$k \leq 0$} (c4r); 
%  \draw [o->,lcnorm] (t4.east) -- (p9);
% -------------------------------------------------
% Finally, the twisty connectors. Again, we place the annotation
% first, then draw the connector
%\path (t1.east) to node [near start, yshift=1em] {$n$} (c1); 
%  \draw [o->,lcfree] (t1.east) -- (c1) |- (p4);
%\path (t2.east) -| node [very near start, yshift=1em] {$n$} (c1); 
%  \draw [o->,lcfree] (t2.east) -| (c1);
%\path (t4.west) to node [yshift=-1em] {$k>0$} (c4); 
%  \draw [*->,lcnorm] (t4.west) -- (c4) |- (p3);
%\path (t5.east) -| node [very near start, yshift=1em] {$n$} (c6); 
%  \draw [o->,lcfree] (t5.east) -| (c6); 
%\path (t6.east) to node [near start, yshift=1em] {$n$} (c6); 
%  \draw [o->,lcfree] (t6.east) -| (c7); 
%\path (t7.east) to node [yshift=-1em] {$k \leq 0$} (c7); 
%  \draw [o->,lcfree] (t7.east) -- (c7)  |- (p9);
%\path (t7.west) to node [yshift=-1em] {$k>0$} (c5); 
%  \draw [*->,lcfree] (t7.west) -- (c5) |- (p5);
% -------------------------------------------------
% A last flourish which breaks all the rules
%\draw [->,purple, dotted, thick, shorten >=1mm]
%  (p9.south) -- ++(5mm,-3mm)  -- ++(27mm,0) 
%  |- node [black, near end, yshift=0.75em, it]
%    {(When message + resources available)} (p0);
% -------------------------------------------------



\end{tikzpicture}
% =================================================
\label{fig:alg}
\caption{Algorithm}
\end{figure}

\end{center}


% subsection adaptive_time_stepping (end)

% chapter algorithm (end)